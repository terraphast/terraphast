C++ tool to check for and enumerate terraces in phylogenetic tree space. 



{\bfseries Usage}\+: {\ttfamily terraces/build/release/app $<$nwk file$>$ $<$gene/site file$>$}

Terraphast takes a .nkw file in Newick format and a genes/sites file, which denotes whether (1) or not (0) gene i is present in species j.

Program output states some imput data properties, the species whose leaf edge is used as a new tree root, and the resulting supertree in compressed newick format.

{\bfseries Compressed Newick Format}\+: The resulting supertree representation cann be plain Newick, but can also contain the following two notation enhancements\+:
\begin{DoxyItemize}
\item {\ttfamily \{a,b,c\}} represents any conceivable binary subtree comprising the taxa a, b, and c.
\item {\ttfamily (A$\vert$B,C$\vert$D)} represents any conceivable binary subtree comprising either subtrees A or B on the left, and either subtrees C or D on the right branch.
\end{DoxyItemize}

Both enhancements were chosen such that the result is standard newick format if there\textquotesingle{}s only one possible supertree.

\subsection*{The Terrace Phenomenon and Problem}

In recent years, it has become common practice to infer phylogenies on so-\/called multi-\/gene datasets. Concatenated multi-\/gene datasets usually exhibit holes, that is, sequence data for some species might not be available for some genes G i in our concatenated dataset. This can be due to a plethora of reasons, for instance, a specific species might simply not have a specific gene G i or the specific gene has simply not been sequenced for some of the species. After concatenating genes (partitions) we therefore end up with an alignment that contains patches of missing data\+:


\begin{DoxyCode}
index       0123

Species 1   AC--
Species 2   AG--
Species 3   ACTT
Species 4   --AG
Species 5   --GG
\end{DoxyCode}


Under the likelihood model conditions that generate terraces, the log likelihood Ln\+L(\+T) of a tree T can be computed as follows\+: Ln\+L(\+T) = LnL(T$\vert$\+G1) + LnL(T$\vert$\+G2) where T$\vert$\+Gi denotes the tree topology induced by T for the species/sequences in partition i for which we have sequence data. In our example, the trees induced by G1 and G2 contain only three taxa. We know that there\textquotesingle{}s only one tree topology with three taxa. On the other hand, there are 15 possible topologys for 5-\/taxa trees. So all 15 possible 5-\/taxon trees for our example dataset will induce the same per-\/gene/partition trees and therefore span a terrace of size 15. This example dataset is bad\+: It does not contain any signal for disentangling the phylogenetic history of these 5 species, since they are only connected via species 3.

{\bfseries Terraces}\+: two distinct comprehensive (containing all n species) trees are on a terrace if all induced per-\/partition subtrees of the two trees are identical. This phenomenon was named and described in \mbox{[}S\+M\+S11\mbox{]}.

Knowing about the phenomenon of terraces, researchers might want to know (i) if a given tree is on a terrace, (ii) how many trees there are on that terrace, and (iii) how the trees on that terrace look like.

\subsection*{The Basic Approach}

TO P\+UT IN H\+E\+RE\+:
\begin{DoxyItemize}
\item Take N\+WK and D\+A\+TA file and re-\/root the tree so that the comprehensive taxon is a leaf under the root, and the rest of the tree is a subtree under the root.
\item Extract constraints according to Constantinescu\textquotesingle{}s algorithm (Only reference and very short outline)
\item Using the constraints, generate trees according to C\textquotesingle{}s algorithm
\item No guarantees for completeness (simply unknown)
\end{DoxyItemize}

\subsection*{A Short Guide to the Code}

This can be found \hyperlink{md_documentation_walkthrough}{here}.

\subsection*{Improvements and Optimizations to the basic approach}

\subsubsection*{Implemented\+:}


\begin{DoxyItemize}
\item We introduced an optional, {\bfseries compressed tree output format}. This format makes printing to terminal faster, since not all possible trees are listed in full detail. See section {\itshape Enhanced Newick Format} above. \mbox{[}\href{https://git.scc.kit.edu/bioinfo2017/terraces/issues/8}{\tt https\+://git.\+scc.\+kit.\+edu/bioinfo2017/terraces/issues/8}\mbox{]}
\item {\bfseries Memory allocation in large blocks}, and managing them with free lists. \mbox{[}\href{https://git.scc.kit.edu/bioinfo2017/terraces/issues/37,}{\tt https\+://git.\+scc.\+kit.\+edu/bioinfo2017/terraces/issues/37,} \href{https://git.scc.kit.edu/bioinfo2017/terraces/issues/13}{\tt https\+://git.\+scc.\+kit.\+edu/bioinfo2017/terraces/issues/13}\mbox{]}
\item {\bfseries Deletion of unnecessary constraints}. \mbox{[}\href{https://git.scc.kit.edu/bioinfo2017/terraces/issues/23}{\tt https\+://git.\+scc.\+kit.\+edu/bioinfo2017/terraces/issues/23}\mbox{]}
\item {\bfseries Improved data structures}\+: We replaced index vectors representing the current leaves by bitvectors with rank support, thus improving space requirements and the efficiency of constraint filtering. The union-\/find data structure could be improved by storing the set ranks in out-\/of-\/bounds indices, thus halving the storage. \mbox{[}\href{https://git.scc.kit.edu/bioinfo2017/terraces/issues/29}{\tt https\+://git.\+scc.\+kit.\+edu/bioinfo2017/terraces/issues/29}\mbox{]}
\item {\bfseries Remap constraints}\+: By removing inner nodes from the constraint numbering, we were able to halve the space requirements of most of our data structures. \mbox{[}\href{https://git.scc.kit.edu/bioinfo2017/terraces/issues/21}{\tt https\+://git.\+scc.\+kit.\+edu/bioinfo2017/terraces/issues/21}\mbox{]}
\item {\bfseries Use specialized bit manipulation instructions}\+: Bipartition iteration and bitvector operations (bit iteration and rank computation) were improved significantly by using specialized C\+PU instructions supported by Compiler intrinsics.
\item {\bfseries Provide a fast terrace check}\+: We discovered that checking for the existence of multiple trees on a terrace can be done without explicitly building any of these trees, thus decreasing the runtime even further.
\item {\bfseries Implemented validation methods}\+: For checking whether the trees generated by our algorithm are indeed distinct and equivalent to the input tree with respect to the missing data, we implemented a fast isomorphy check operating directly on our data structures.
\item Since we wanted to implement different versions of the algorithm, we used a {\bfseries generic enumerator} which relies on callback methods to implement the concrete version of the algorithm (terrace checking, tree counting or multitree construction). This method also allows us to attach {\bfseries logging and status update decorators} to the algorithm to check the internal computations or monitor the progress of the algorithm.
\item Short of support for arbitrary-\/precision math, our implementation is $\ast$$\ast$fully compatible with Visual C++$\ast$$\ast$ in addition to the normal gcc/clang support.
\end{DoxyItemize}

\subsubsection*{Planned\+:}


\begin{DoxyItemize}
\item Enumerate subtrees in {\bfseries parallel}. One challenge would be separation of the workload so that multiple threads have \char`\"{}enough to do\char`\"{}. Another challenge would be the merging of the individual threads\textquotesingle{} results. \mbox{[}\href{https://git.scc.kit.edu/bioinfo2017/terraces/issues/6}{\tt https\+://git.\+scc.\+kit.\+edu/bioinfo2017/terraces/issues/6}\mbox{]}
\item Finding {\bfseries good heuristics} for choosing a subtree into which we want to descend first. Ideally, we\textquotesingle{}d have a nice heuristic that tells us which subtrees and associated constraints probably give us several options to construct a supertree, in which case we can safely answer \char`\"{}yes\char`\"{} to the question \char`\"{}are we on a terrace?\char`\"{}. {\bfseries Ideas}\+: \char`\"{}\+Smallest subtree first\char`\"{}, \char`\"{}\+Least constraints first\char`\"{}. \mbox{[}\href{https://git.scc.kit.edu/bioinfo2017/terraces/issues/3}{\tt https\+://git.\+scc.\+kit.\+edu/bioinfo2017/terraces/issues/3}\mbox{]}
\item ...
\end{DoxyItemize}

\subsection*{References}

\mbox{[}S\+M\+S11\mbox{]} Michael J Sanderson, Michelle M Mc\+Mahon, and Mike Steel. Terraces in phylogenetic tree space. Science, 333(6041)\+:448–450, 2011. 